\documentclass[10pt]{beamer}

\usetheme{metropolis}
\usepackage{appendixnumberbeamer}

\usepackage{booktabs}
\usepackage[scale=2]{ccicons}
\usepackage{graphicx}
\usepackage{hyperref}
\usepackage{pgfplots}
\usepgfplotslibrary{dateplot}

\usepackage{xspace}
\newcommand{\themename}{\textbf{\textsc{metropolis}}\xspace}

\title{An Introduction to Singular Value Decomposition}
\subtitle{Principle and Applications}
\date{\today}
\author{Pei Liu}
\institute{Department of Computer Science @UESTC}
% \titlegraphic{\hfill\includegraphics[height=1.5cm]{logo.pdf}}

\begin{document}

\maketitle

\begin{frame}{Table of contents}
  \setbeamertemplate{section in toc}[sections numbered]
  \tableofcontents[hideallsubsections]
\end{frame}

\section{Review of basics}

\begin{frame}{Review - Eigenvalues}
  If $A$ is a symmetric real $n\times n$ matrix, $x$ is a $n$-dimensional vector, $\lambda$ is a real number, and $$Ax=\lambda x$$, then we have:
  \begin{itemize}
    \item $x$ is one of eigenvectors of matrix $A$
    \item $\lambda$ is one of eigenvalues of matrix $A$
  \end{itemize}

  The eigenvectors and eigenvalues are used in \alert{Matrix Decomposition}, which can be expressed as follows: $$A=W\Sigma W^{-1}$$
  where $W$ and $\Sigma$ consist of eigenvectors and eigenvalues.
\end{frame}

\begin{frame}{Review - Eigenvalues}
  How can we utilize the eigenvalues and eigenvectors?

  It's fundamental of many theory. Include but not limit to:
  \begin{itemize}
    \item Matrix diagonalize
    \item PCA dimension reduction
    \item Solution to Markov process
  \end{itemize}

  Here we dive into its function in \textbf{SVD (Singular Value Decomposition)}.
\end{frame}

\section{Singular Value Decomposition}

\begin{frame}{SVD - Definition}
  If $A$ is $m \times n$ matrix, the SVD of matrix $A$ is:
  $$A=U\Sigma V^{T}$$
  where $U^{T}U=I, V^{T}V=I$.

  \begin{figure}[htbp]
    \centering
    \includegraphics[width=0.95\linewidth]{fig/SVD-p1.png}
    \caption{Definition of SVD}
  \end{figure}
\end{frame}

\begin{frame}[fragile]{SVD - Solution}
  How to obtain the value of $U, V, \Sigma$ ?

  With the help of eigenvalues and eigenvectors, it is easy for us to construct matrixs $A^{T}A$ and $AA^{T}$ to get the solution.
\end{frame}

\section{Examples}

\begin{frame}[fragile]{Examples of SVD}
  If $$\mathbf{A} = \left( \begin{array}{ccc} 0& 1\\  1& 1\\   1& 0 \end{array} \right)$$
  We can construct matrixs $A^{T}A$ and $AA^{T}$ by $$\mathbf{A^TA} = \left( \begin{array}{ccc} 0& 1 &1\\ 1&1& 0 \end{array} \right) \left( \begin{array}{ccc} 0& 1\\  1& 1\\   1& 0 \end{array} \right) = \left( \begin{array}{ccc} 2& 1 \\ 1& 2 \end{array} \right)$$
  and $$\mathbf{AA^T} =  \left( \begin{array}{ccc} 0& 1\\  1& 1\\   1& 0 \end{array} \right) \left( \begin{array}{ccc} 0& 1 &1\\ 1&1& 0 \end{array} \right) = \left( \begin{array}{ccc} 1& 1 & 0\\ 1& 2 & 1\\ 0& 1& 1 \end{array} \right)$$
  then we find the eigenvectors and eigenvalues of $\mathbf{A^TA}$:
  $$\lambda_1= 3; v_1 = \left( \begin{array}{ccc} 1/\sqrt{2} \\ 1/\sqrt{2} \end{array} \right); \lambda_2= 1; v_2 = \left( \begin{array}{ccc} -1/\sqrt{2} \\ 1/\sqrt{2} \end{array} \right)$$
\end{frame}

\begin{frame}[fragile]{Examples of SVD}
  the eigenvectors and eigenvalues of $\mathbf{AA^T}$:
  $$\lambda_1= 3; u_1 = \left( \begin{array}{ccc} 1/\sqrt{6} \\ 2/\sqrt{6} \\ 1/\sqrt{6} \end{array} \right); \lambda_2= 1; u_2 = \left( \begin{array}{ccc} 1/\sqrt{2} \\ 0 \\ -1/\sqrt{2} \end{array} \right);  \lambda_3= 0; u_3 = \left( \begin{array}{ccc} 1/\sqrt{3} \\ -1/\sqrt{3} \\ 1/\sqrt{3} \end{array} \right)$$
  
  Then we can find the singular values using the equation $\sigma_i = \sqrt{\lambda_i}$, so the singular values are expressed by: $$\sigma_1 = \sqrt{3}, \sigma_2=1$$

  As a result, \textbf{Singular Value Decomposition} of matrix $A$ can be summarized as:
  $$U\Sigma V^T = \left( \begin{array}{ccc} 1/\sqrt{6} & 1/\sqrt{2} & 1/\sqrt{3} \\ 2/\sqrt{6} & 0 & -1/\sqrt{3}\\ 1/\sqrt{6} & -1/\sqrt{2} & 1/\sqrt{3} \end{array} \right) \left( \begin{array}{ccc} \sqrt{3} & 0 \\  0 & 1\\ 0 & 0 \end{array} \right) \left( \begin{array}{ccc} 1/\sqrt{2}  & 1/\sqrt{2}  \\ -1/\sqrt{2}  & 1/\sqrt{2}  \end{array} \right)$$
\end{frame}

\section{Applications}

\begin{frame}{Applications - Important feature extraction}
  We first to arrange the singular values in descending order, then extract the top-$k$ the singular values:
  $$A_{m \times n} = U_{m \times m}\Sigma_{m \times n} V^T_{n \times n} \approx U_{m \times k}\Sigma_{k \times k} V^T_{k \times n}$$
  \begin{figure}[htbp]
    \centering
    \includegraphics[width=0.95\linewidth]{fig/SVD-p2.png}
    \caption{Applications of SVD}
  \end{figure}
\end{frame}

\begin{frame}{Applications - Important feature extraction}
  One of application scenario is:
  \begin{figure}[htbp]
    \centering
    \includegraphics[width=0.45\linewidth]{fig/SVD-p3.png}
    \caption{Important feature extraction by SVD}
  \end{figure}
\end{frame}

\begin{frame}{Applications - More}
  \begin{itemize}
    \item Data compression
    \item Noise reduction (or feature extraction)
    \item PCA dimension reduction
  \end{itemize}
\end{frame}

\begin{frame}[standout]
  Questions?
\end{frame}

\begin{frame}{References}
  References:
  \begin{itemize}
    \item \href{http://blog.codinglabs.org/articles/pca-tutorial.html}{PCA - Tutorial}
    \item \href{http://www.ams.org/publicoutreach/feature-column/fcarc-svd}{We Recommend a Singular Value Decomposition}
    \item \href{https://www.cnblogs.com/pinard/p/6251584.html}{Cnblog - tutorial of SVD}
  \end{itemize}
\end{frame}

\end{document}