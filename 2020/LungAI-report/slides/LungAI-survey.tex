\documentclass[10pt]{beamer}

\usetheme{metropolis}
\usepackage{appendixnumberbeamer}

\usepackage{booktabs}
\usepackage[scale=2]{ccicons}
\usepackage{graphicx}
\usepackage{hyperref}
\usepackage{pgfplots}
\usepgfplotslibrary{dateplot}

\usepackage{xspace}
\newcommand{\themename}{\textbf{\textsc{metropolis}}\xspace}
\usepackage{xeCJK}
\setmainfont{Times New Roman}
\setCJKmainfont[BoldFont = STZhongsong, ItalicFont = STKaiti]{STSong}

\title{肺癌的人工智能研究}
\subtitle{CT图像应用}
\date{汇报日期:2020年8月1日}
\author{汇报人:刘沛}
\institute{四川大学华西医院}

\begin{document}

\maketitle

\begin{frame}{目录}
  \setbeamertemplate{section in toc}[sections numbered]
  \tableofcontents[hideallsubsections]
\end{frame}

\section{背景与意义}

\begin{frame}{肺癌相关背景}
  肺癌 (Lung Cancer)
  \begin{itemize}
    \item \textit{威胁最大的恶性肿瘤之一}
    \item \textit{发病率和死亡率很高,增长速度最快}
  \end{itemize}

  降低肺癌死亡率:低剂量CT早期筛查\\
  \vbox{}
  \begin{quote}
    National Lung Screening Trial Research Team. Reduced Lung-Cancer Mortality with Low-Dose Computed Tomographic Screening. N. Engl. J. Med. 2011, 365, 395–409.
  \end{quote}
  
\end{frame}

\begin{frame}{计算辅助诊断}
  \begin{figure}[H]
    \centering
    \includegraphics[width=0.7\linewidth]{fig/S1-1.pdf}
    \caption{计算机辅助诊断}
  \end{figure}
\end{frame}

\section{实际应用}

\begin{frame}{实际应用介绍}
  基于CT图像的人工智能应用:
  \begin{itemize}
    \item A1: \textit{肺结节识别}
    \item A2: \textit{肺结节检测}
    \item A3: \textit{肺部语义分割}
    \item A4: \textit{肺癌风险预测}
  \end{itemize}
\end{frame}

\begin{frame}{应用介绍步骤}
  主要从以下方面介绍:
  \begin{itemize}
    \item \textit{目的是什么?}
    \item \textit{如何达到目的?(流程简介)}
    \item \textit{存在哪些难点?}
  \end{itemize}
\end{frame}

\begin{frame}{A1: 肺结节识别}
  识别肺结节属于哪一种已知的类别,如
  \begin{itemize}
    \item \textit{肺结节良恶性识别}
    \item \textit{肺纹理特征识别}
  \end{itemize}
  \begin{columns}[T,onlytextwidth]
    \column{0.5\textwidth}
    \begin{figure}[htbp]
      \centering
      \includegraphics[width=0.95\linewidth]{fig/S2-2.png}   
      \caption{肺结节良恶性}
    \end{figure}
    \column{0.5\textwidth}
    \begin{figure}[htbp]
      \centering
      \includegraphics[width=0.95\linewidth]{fig/S2-1.png}
      \caption{肺纹理特征}
    \end{figure}
  \end{columns}
\end{frame}

\begin{frame}{A1: 肺结节识别}
  肺结节识别模型:深度卷积神经网络模型 (DCNN)
  \begin{itemize}
    \item 输入:\textit{肺结节/肺部候选区域 (RoI, Region of Interst) CT图像 (2D/3D)}
    \item 输出:\textit{RoI所属类别概率}
  \end{itemize}
  \begin{figure}[htbp]
      \centering
      \includegraphics[width=0.95\linewidth]{fig/S2-3.png}
      \caption{肺结节识别模型}
  \end{figure}
\end{frame}

\begin{frame}{A1: 肺结节识别}
  计算机视觉\textbf{图像分类}任务
  \vspace{5 mm}

  任务流程:
  \begin{itemize}
    \item 数据收集及预处理:\textit{带有标签的RoI CT图像}
    \item 模型训练:\textit{数据拟合,具备图像特征提取能力}
    \item 模型测试与评估:\textit{ACC,AUC、F1等分类模型常用指标}
  \end{itemize}
\end{frame}

\begin{frame}{A1: 肺结节识别}
  计算机视觉\textbf{图像分类}任务
  \vspace{5 mm}

  难点:
  \begin{itemize}
    \item \textit{大型,高质量标注数据集}
    \item \textit{正负样本比例不平衡}
    \item \textit{模型本身表达能力不够}
  \end{itemize}
\end{frame}

\begin{frame}{A2: 肺结节检测}
  计算机视觉\textbf{目标检测}任务
  \begin{itemize}
    \item \textbf{定位}:\textit{自动定位肺结节所在区域}
    \item \textbf{识别}:\textit{自动识别所属类别(如是/不是结节)}
  \end{itemize}

  \begin{figure}[htbp]
      \centering
      \includegraphics[width=0.98\linewidth]{fig/S2-4.png}
      \caption{肺结节检测模型}
  \end{figure}
\end{frame}

\begin{frame}{A2: 肺结节检测}
  常用的检测框架,两个子任务:
  \begin{itemize}
    \item \textbf{候选区域检测}:\textit{从CT图像中检测所有可能的结节区域,即RoI}
    \item \textbf{降低误报率}:\textit{识别RoI是否包含结节,即False Positive Reduction}
  \end{itemize}

  训练数据:带有肺结节\textbf{边界框标注}的CT图像

  \begin{figure}[htbp]
      \centering
      \includegraphics[width=0.98\linewidth]{fig/S2-4.png}
      \caption{肺结节检测模型}
  \end{figure}
\end{frame}

\begin{frame}{A2: 肺结节检测}
  其它的检测框架:
  \begin{itemize}
    \item \textit{One-Stage方法,同时完成检测和识别}
    \item \textit{基于3D图像的神经网络模型}
    \item \textit{不同的图片预处理或训练策略}
    \item \textit{弱监督学习检测}
  \end{itemize}

  难点:
  \begin{itemize}
    \item \textit{数据标注和类别不平衡}
    \item \textit{高质量的RoI}
    \item \textit{False Positive Reduction}
    \item \textit{3D 图像}
  \end{itemize}
\end{frame}

\begin{frame}{A3: 肺部语义分割}
  计算机视觉\textbf{语义分割}任务:\textit{图像每个像素点所属类别}

  肺部语义分割:从CT图像中分割出感兴趣的实例
   
  \begin{figure}[htbp]
    \centering
    \includegraphics[width=0.98\linewidth]{fig/S2-5.png}
    \caption{肺实质分割模型 (2D/3D)}
  \end{figure}

  分割结果一般用于更加精细的下游任务,如肺结节检测等

\end{frame}

\begin{frame}{A3: 肺部语义分割}
  计算机视觉\textbf{语义分割}任务:\textit{图像每个像素点所属类别}

  肺部语义分割:从CT图像中分割出感兴趣的实例

  \begin{figure}[htbp]
    \centering
    \includegraphics[width=0.85\linewidth]{fig/S2-6.png}
    \caption{肺结节分割模型 (2D/3D)}
  \end{figure}
\end{frame}

\begin{frame}{A3: 肺部语义分割}
  肺部语义分割方法:
  \begin{itemize}
    \item \textit{区域生长法}
    \item \textit{基于阈值的多阶段方法}
    \item \textit{3D Mask R-CNN实例分割网络}
    \item \textit{其他半监督学习方法}
  \end{itemize}
\end{frame}

\begin{frame}{A3: 肺部语义分割}
  Mask R-CNN:像素到像素的Mask预测

  训练数据:带有感兴趣实例\textbf{像素级别}标注的图像
  \begin{figure}[htbp]
    \centering
    \includegraphics[width=0.9\linewidth]{fig/S2-7.png}
    \caption{Mask R-CNN模型}
  \end{figure}
\end{frame}

\begin{frame}{A3: 肺部语义分割}
  难点:
  \begin{itemize}
    \item \textit{大规模的肺部CT图像语义标注数据}
    \item \textit{更加精细的3D图像分割结果}
  \end{itemize}
\end{frame}

\begin{frame}{A4: 肺癌风险预测}
  \begin{quote}
    如何直接使用Chest CT扫描图像,预测患肺癌的风险?
    \vspace{5 mm}

    End-to-end lung cancer screening with three-dimensional deep learning on low-dose chest computed tomography, Nature Medicine, 2019.
  \end{quote}
\end{frame}

\begin{frame}{A4: 肺癌风险预测}
  CT图像上端到端的肺癌筛查深度学习模型

  \begin{figure}[htbp]
    \centering
    \includegraphics[width=0.9\linewidth]{fig/S2-8.png}
    \caption{Nature Medicine(2019): End-to-end lung cancer screening with three-dimensional deep learning on low-dose chest computed tomography.}
  \end{figure}
\end{frame}

\section{核心方法}

\begin{frame}{核心方法介绍}
  CT图像分析关键方法 / 架构:
  \begin{itemize}
    \item N1: \textit{卷积神经网络}
    \item N2: \textit{目标识别网络}
    \item N3: \textit{目标检测网络}
    \item N4: \textit{语义分割网络}
    \item N5: \textit{多任务网络}
  \end{itemize}
\end{frame}

\begin{frame}{方法介绍步骤}
  主要从以下方面介绍:
  \begin{itemize}
    \item \textit{基本架构}
    \item \textit{用途,特点,及优势}
  \end{itemize}
\end{frame}

\begin{frame}{N1: 卷积神经网络基础}
  卷积神经网络 (CNN, Convolutional Neural Network) 是如何“学习”的
  \begin{itemize}
    \item 正向传播:\textit{计算图像特征、预测结果、预测误差}
    \item 反向传播:\textit{计算误差梯度,更新网络参数}
  \end{itemize}
  \begin{figure}[htbp]
    \centering
    \includegraphics[width=0.8\linewidth]{fig/S3-1.png}
    \caption{经典的卷积神经网络:VGGNet (2014)}
  \end{figure}
\end{frame}

\begin{frame}{N1: 卷积神经网络基础}
  卷积神经网络 (CNN, Convolutional Neural Network) 学习到了什么?
  \begin{figure}[H]
    \centering
    \includegraphics[width=0.95\linewidth]{fig/S3-2.png}
    \caption{卷积神经网络可视化}
  \end{figure}
\end{frame}

\begin{frame}{N1: 目标识别网络}
  \begin{figure}[H]
    \centering
    \includegraphics[width=0.5\linewidth]{fig/S3-3.png}
    \caption{经典的图像分类网络:ResNet (ILSVRC-2015 Rank \#1)}
  \end{figure}
\end{frame}

\begin{frame}{N1: 目标识别网络}
  肺结节识别应用中,常用的CNN模型有
  \begin{itemize}
    \item \textit{ResNet, DenseNet, SENet}
    \item \textit{Inception}
  \end{itemize}

  特点及优势:
  \begin{itemize}
    \item \textit{强大的表达能力}
    \item \textit{适用于不同任务}
  \end{itemize}
\end{frame}

\begin{frame}{N2: 目标检测网络}
  Alibaba’s 2017 TianChi AI Competition for Healthcare Rank \#1

  \begin{figure}[H]
    \centering
    \includegraphics[width=0.7\linewidth]{fig/S3-4.png}
    \caption{候选框生成: 3D Faster R-CNN}
  \end{figure}
  \begin{figure}[H]
    \centering
    \includegraphics[width=0.6\linewidth]{fig/S3-5.png}
    \caption{降低FP: 3D DCNN}
  \end{figure}
\end{frame}

\begin{frame}{N2: 目标检测网络}
  CV领域经典的目标检测网络:
  \begin{itemize}
    \item \textit{Faster R-CNN}
    \item \textit{FPN, RetinaNet}
    \item \textit{YOLO}
  \end{itemize}
  \vspace{5 mm}
  \begin{quote}
    [1] MICCAI (2019). 3DFPN-HS2: 3D Feature Pyramid Network Based High Sensitivity and Specificity Pulmonary Nodule Detection. 

    [2] In Proceedings of the Medical Imaging (2018). Using YOLO Based Deep Learning Network for Real Time Detection and Localization of Lung Nodules from Low Dose CT Scans.

    [3] Nature Medicine (2019). End-to-end lung cancer screening with three-dimensional deep learning on low-dose chest computed tomography. 
  \end{quote}
\end{frame}

\begin{frame}{N3: 语义分割网络}
  语义分割网络 (Segmentation in a pixel-to-pixel manner):
  \begin{itemize}
    \item \textit{U-Net, U-Net++}
    \item \textit{Mask R-CNN}
    \item \textit{DeepLab}
  \end{itemize}

  \begin{figure}
    \centering
    \includegraphics[width=0.8\linewidth]{fig/S3-6.png}
    \caption{U-Net for medical image segmentation}
  \end{figure}
\end{frame}

\begin{frame}{N4: 多任务网络}
  多任务深度学习模型: \textit{Segmentation + Detection + Classification}
  \begin{figure}
    \centering
    \includegraphics[width=\linewidth]{fig/S3-7.png}
    \caption{End-to-end cancer risk prediction model}
  \end{figure}
\end{frame}

\section{发展方向}

\begin{frame}{医学图像处理方法发展方向}
  参考文献:近年发表在医学图像处理会议或期刊上的部分文章
  \vspace{5 mm}

  从\textbf{卷积神经网络模型}层面:
  \begin{enumerate}[(1)]
    \item 借鉴计算机视觉领域最新的模型架构,用于特定任务
    \item 综合各个网络的优势及特点,设计用于特定任务的神经网络
    \item 适用于3D图像的网络结构
  \end{enumerate}

\end{frame}

\begin{frame}{医学图像处理方法发展方向}
  参考文献:近年发表在医学图像处理会议或期刊上的部分文章
  \vspace{5 mm}

  从\textbf{输入模型的数据}层面:
  \begin{enumerate}[(1)]
    \item 针对标注数据过少的问题,设计半监督学习或者自监督学习的方法训练模型
    \item 针对数据正负样本不均衡的问题,设计新的目标函数减小少数样本的稀疏性
  \end{enumerate}

  \textbf{肺癌预后生存分析?}
\end{frame}

\begin{frame}[standout]
  谢谢各位老师
\end{frame}

\end{document}